\documentclass[12pt,a4paper]{article}
% Change "article" to "report" to get rid of page number on title page
\usepackage{amsmath,mathtools,amsfonts,amsthm,amssymb}
\usepackage{setspace}
\usepackage{Tabbing}
\usepackage{cite}
\usepackage{fancyhdr}
\usepackage{lastpage}
\usepackage{extramarks}
\usepackage{chngpage}
\usepackage{fourier}
\usepackage{soul,color}
\usepackage[usenames,dvipsnames]{xcolor}
\usepackage{graphicx,float,wrapfig}
\usepackage[utf8]{inputenc}
\usepackage{sidecap}
\usepackage{marvosym}
\usepackage{tikz, tikz-qtree}
\usepackage{tabularx, multirow}
\usepackage{enumerate}
\usepackage{hyperref}
\usepackage{cleveref}
\usepackage{zref}
\usepackage{pgffor}
\definecolor{gray99}{gray}{.99}
\usepackage{listings}
\usepackage[english]{babel}
\usepackage{placeins}
\usepackage{tikz}
\usepackage{tikz-qtree}
\usepackage{xspace}
\usepackage{mathtools}
\usepackage{tabulary}
\lstset{
	language=Matlab,
	backgroundcolor=\color{gray99},
	tabsize=3,
	frame=single,
	keywordstyle=\ttfamily\bfseries\color{RoyalBlue},
	commentstyle=\ttfamily\color{ForestGreen},
	stringstyle=\ttfamily\color{Gray},
	breaklines=true,
	showstringspaces=false,
	basicstyle=\small\ttfamily,
	emph={label},
	xleftmargin=22pt,
	framexleftmargin=22pt,
	framexrightmargin=0pt,
	framexbottommargin=4pt,
	numbers=left,
	stepnumber=1
}
\usepackage{caption}
\DeclareCaptionFont{black}{\color{black}}{\bfseries}
\DeclareCaptionFormat{listing}{\parbox{\textwidth}{\hspace{8pt}#1#2#3}}
\captionsetup[lstlisting]{format=listing,labelfont=black,textfont=black, singlelinecheck=false, margin=0pt, font={bf,footnotesize}}

% In case you need to adjust margins:
\topmargin=-0.45in      %
\evensidemargin=0in     %
\oddsidemargin=0in      %
\textwidth=6.5in        %
\textheight=9.5in       %
\headsep=0.25in         %

% Homework Specific Information
\newcommand{\hmwkTopic}{Automatic Material Recognition}
\newcommand{\hmwkTitle}{\hmwkTopic}
\newcommand{\hmwkDueDate}{\today}
\newcommand{\hmwkClass}{CS 499: Senior Thesis}
\newcommand{\hmwkAuthorNameA}{David Alexander Forsyth}
\newcommand{\hmwkAuthorEmailA}{daf@illinois.edu}
\newcommand{\hmwkAuthorNameB}{José Vicente Ruiz Cepeda}
\newcommand{\hmwkAuthorEmailB}{ruizcep2@illinois.edu}

% Setup the header and footer
\pagestyle{fancy}                                                       %
\lhead{\hmwkAuthorNameA}                                                 %
\chead{CS 499}  %
\rhead{\hmwkTopic}     
                                                %
\lfoot{}                                                %
\cfoot{\hyperlink{toc}{\thepage}}%
\rfoot{}                          %

\renewcommand\headrulewidth{0.4pt}                                      %
\renewcommand\footrulewidth{0.4pt}                                      %


%%%%%%%%%%%%%%%%%%%%%%%%%%%%%%%%%%%%%%%%%%%%%%%%%%%%%%%%%%%%%
% Make title
\title{\vspace{2in}\textmd{\hmwkClass\\\textbf{\hmwkTitle}}\\\normalsize\vspace{0.1in}\small{\hmwkDueDate}\\\vspace{4in}}
\date{}
\author{
\textbf{\hmwkAuthorNameA} <\texttt{\href{mailto:ruizcep2@illinois.edu}{\hmwkAuthorEmailA}}> \\
\textbf{\hmwkAuthorNameB} <\texttt{\href{mailto:ruizcep2@illinois.edu}{\hmwkAuthorEmailB}}>}
%%%%%%%%%%%%%%%%%%%%%%%%%%%%%%%%%%%%%%%%%%%%%%%%%%%%%%%%%%%%%

\begin{document}
\begin{singlespace}

\begin{titlepage}
\maketitle
\thispagestyle{empty}
\end{titlepage}

% Uncomment the \tableofcontents and \newpage lines to get a Contents page
% Uncomment the \setcounter line as well if you do NOT want subsections
%       listed in Contents
%\setcounter{tocdepth}{1}
\hypertarget{toc}{}
\tableofcontents
\newpage

% When problems are long, it may be desirable to put a \newpage or a
% \clearpage before each homeworkProblem environment


\clearpage

% Own commands:
\newcommand{\e}[1]{\emph{#1}\xspace}
\crefformat{footnote}{#2\footnotemark[#1]#3} % Required for footnotes with href.

%%%%%%%%
\section{Introduction}

The word \emph{material} has its origin in the \emph{Latin} word `\emph{materia}', in English, \emph{matter}. In the same way that reality is made of \emph{matter}, every concrete entity can be defined in terms of materials: the silk of a dress, the scales of a python or the concrete of the pavement, for example. Therefore, materials represent an important part of the real world, since they are the key to understand and infer the physical properties of objects just looking at them. \\

Human beings, thanks to thousand of years of evolution, have become very good at recognizing materials. This skill could even determine the difference between life and death in the beginning of times, when being able to start a fire depended on choosing the right type of rock to make sparks. Nowadays, fortunately, our life does not depend on this ability, but we still use it a lot, even without noticing, for example, when deciding which side of a sidewalk step to avoid slippery ice during winter. \\

Because of all of this, an automatic system able to see the objects and recognize their materials would be really valuable, since it would be one more step on the way of building a machine able to see as well as a human, or even better. In that sense, despite of the achievements of \emph{Computer Vision}, material recognition is still an open problem in the field. \\

We present a new approach in the problem of materials recognition, using feature extraction methods and \emph{Support Vector Machines} (\emph{SVMs}) to obtain the physical properties of the objects and applying \emph{SVMs} again to this intermediate dataset to obtain the final material category in a set of 18 possible ones. The obtained results using the dataset of \cite{Liao_2013_CVPR} beat any other system used on this dataset and prove that this approach has a great potential as a first step in the solution to the problem.

\vspace{2cm}
--- In construction ---

%%%%%%%%
\section{Dataset}
This first assignment consists on the markup of a material dataset with several properties. The dataset comes from~\cite{Liao_2013_CVPR} and is composed of eighteen material categories, each one with twelve different sample images in \texttt{PNG} format with resolutions lower than 400 x 400 pixels.

The procedure to mark each of the material images consists on defining its shape and touch for the \e{fine}, \e{medium} and \e{coarse} scale, being its possible values the following ones:

\section{Approach}

\section{Results}

\section{Conclusion}

% Define commands to easy the classfication. First letter property, second and third letters value.
\newcommand{\Fine}{\textbf{Fine}}
\newcommand{\Medium}{\textbf{Medium}}
\newcommand{\Coarse}{\textbf{Coarse}}

\newcommand{\sfl}{\e{flat}}
\newcommand{\sro}{\e{round}}
\newcommand{\sexor}{\e{extended organized}}
\newcommand{\sexdi}{\e{extended disorganized}}

\newcommand{\tfu}{\e{furry}} 
\newcommand{\tfe}{\e{feathery}} 
\newcommand{\tco}{\e{coarse}} 
\newcommand{\tro}{\e{rough}} 
\newcommand{\tbu}{\e{bumpy}} 
\newcommand{\tsc}{\e{scratchy}} 
\newcommand{\tsm}{\e{smooth}} 
\newcommand{\tve}{\e{velvety}}

\begin{itemize}
	\item \textbf{Shape}: \sfl\footnote{\label{new}New value.}, \sro, \sexor and \sexdi.
	\item \textbf{Touch}: \tfu, \tfe, \tco, \tro\cref{new}, \tbu, \tsc, \tsm and \tve\cref{new}.
\end{itemize}

\clearpage
\bibliography{final_bibliography}
\bibliographystyle{abbrv}

\begin{appendix}
    %%!TEX root = final_report.tex

%%%%%%%%
\section{Markup}

% Markup commands.
\newcommand{\imageSize}{0.30}
\newcommand{\dataSize}{0.65}
\newcommand{\spaceSize}{0.02}

\newcommand{\mat}{Birch}
\newcommand{\Number}{01}
\newcommand{\InputImage}[8]{
\begin{figure}[!ht]
    \centering
    \begin{minipage}{\imageSize\textwidth}
        \centering
        \includegraphics[width=\textwidth]{../../Dataset/\mat/\Number.png}
    \end{minipage}
    \hspace{\spaceSize\textwidth}
    \begin{minipage}{\dataSize\textwidth}
        \textbf{\large \mat \xspace \Number}:

        #7
        \begin{itemize}
            \item \Fine: #1, #2.
            \item \Medium: #3, #4.
            \item \Coarse: #5, #6.
        \end{itemize}
        #8
    \end{minipage}
\end{figure}
\FloatBarrier    
}

%%%
\renewcommand{\mat}{Birch}
\subsection{\mat}
Coarse scale is considered the picture itself, while medium scale is a part where the horizontal pattern is not present, and fine scale a little spot in the medium scale.

\renewcommand{\Number}{01}\InputImage{\sfl}{\tco}{\sexdi}{\tsc}{\sexor}{\tsc}
{}{}
\renewcommand{\Number}{02}\InputImage{\sfl}{\tco}{\sfl}{\tco}{\sexdi}{\tsc}
{}{}
\renewcommand{\Number}{03}\InputImage{\sfl}{\tco}{\sexor}{\tsc}{\sexor}{\tsc}
{}{}
\renewcommand{\Number}{04}\InputImage{\sfl}{\tco}{\sexor}{\tsc}{\sro}{\tco}
{Watermark and big crack.}{}
\renewcommand{\Number}{05}\InputImage{\sfl}{\tco}{\sfl}{\tco}{\sexor}{\tsc}
{}{Considering for the medium scale the part without horizontal lines.}
\renewcommand{\Number}{06}\InputImage{\sfl}{\tco}{\sexor}{\tsc}{\sexor}{\tsc}
{}{}
\renewcommand{\Number}{07}\InputImage{\sfl}{\tco}{\sexor}{\tco}{\sexor}{\tco}
{}{}
\renewcommand{\Number}{08}\InputImage{\sfl}{\tco}{\sexor}{\tsc}{\sexor}{\tsc}
{Watermark.}{}
\renewcommand{\Number}{09}\InputImage{\sfl}{\tco}{\sexor}{\tsc}{\sexdi}{\tsc}
{}{}
\renewcommand{\Number}{10}\InputImage{\sfl}{\tco}{\sexor}{\tsc}{\sexor}{\tsc}
{}{}
\renewcommand{\Number}{11}\InputImage{\sfl}{\tco}{\sexor}{\tsc}{\sro}{\tsc}
{Big crack.}{}
\renewcommand{\Number}{12}\InputImage{\sfl}{\tco}{\sexor}{\tsc}{\sexor}{\tsc}
{}{}

%%%
\clearpage
\renewcommand{\mat}{Brick}
\subsection{\mat}

\renewcommand{\Number}{01}\InputImage{\sfl}{\tco}{\sexor}{\tco}{\sexor}{\tco}
{}{}
\renewcommand{\Number}{02}\InputImage{\sfl}{\tsc}{\sexor}{\tco}{\sexor}{\tco}
{}{Considering as fine scale the mortar.}
\renewcommand{\Number}{03}\InputImage{\sfl}{\tsc}{\sfl}{\tco}{\sexor}{\tsc}
{}{Considering as fine scale the mortar.}
\renewcommand{\Number}{04}\InputImage{\sfl}{\tco}{\sexor}{\tco}{\sexor}{\tco}
{}{}
\renewcommand{\Number}{05}\InputImage{\sfl}{\tco}{\sexor}{\tco}{\sexor}{\tco}
{}{Might also be considered bumpy at coarse and medium scale?}
\renewcommand{\Number}{06}\InputImage{\sfl}{\tco}{\sexor}{\tsm}{\sexor}{\tsm}
{Graffiti.}{Considering as fine scale the mortar.}
\renewcommand{\Number}{07}\InputImage{\sfl}{\tco}{\sexor}{\tco}{\sexor}{\tco}
{}{}
\renewcommand{\Number}{08}\InputImage{\sfl}{\tco}{\sexor}{\tbu}{\sexor}{\tbu}
{}{}
\renewcommand{\Number}{09}\InputImage{\sfl}{\tco}{\sexor}{\tco}{\sexor}{\tco}
{}{}
\renewcommand{\Number}{10}\InputImage{\sfl}{\tsc}{\sexor}{\tco}{\sexor}{\tco}
{}{Considering as fine scale the mortar.}
\renewcommand{\Number}{11}\InputImage{\sexor}{\tco}{\sexor}{\tco}{\sexor}{\tco}
{}{}
\renewcommand{\Number}{12}\InputImage{\sfl}{\tsm}{\sexor}{\tsm}{\sexor}{\tbu}
{}{}

%%%
\clearpage
\renewcommand{\mat}{Concrete}
\subsection{\mat}

\renewcommand{\Number}{01}\InputImage{\sfl}{\tco}{\sfl}{\tco}{\sfl}{\tco}
{}{}
\renewcommand{\Number}{02}\InputImage{\sfl}{\tco}{\sfl}{\tco}{\sexdi}{\tco}
{}{}
\renewcommand{\Number}{03}\InputImage{\sfl}{\tco}{\sfl}{\tco}{\sexdi}{\tbu}
{}{}
\renewcommand{\Number}{04}\InputImage{\sfl}{\tsc}{\sfl}{\tsc}{\sfl}{\tsc}
{}{}
\renewcommand{\Number}{05}\InputImage{\sfl}{\tco}{\sfl}{\tco}{\sexor}{\tco}
{}{}
\renewcommand{\Number}{06}\InputImage{\sfl}{\tco}{\sfl}{\tco}{\sexor}{\tco}
{}{Considering that the almost straight break is organized.}
\renewcommand{\Number}{07}\InputImage{\sfl}{\tco}{\sfl}{\tco}{\sro}{\tsc}
{}{Considering the spots of different colours to determine a round shape.}
\renewcommand{\Number}{08}\InputImage{\sfl}{\tco}{\sfl}{\tco}{\sfl}{\tco}
{}{Considering that the patterns in the middle of the image don't affect the shape.}
\renewcommand{\Number}{09}\InputImage{\sfl}{\tco}{\sfl}{\tco}{\sfl}{\tco}
{}{}
\renewcommand{\Number}{10}\InputImage{\sfl}{\tco}{\sfl}{\tco}{\sexdi}{\tco}
{}{}
\renewcommand{\Number}{11}\InputImage{\sfl}{\tco}{\sfl}{\tsc}{\sfl}{\tsc}
{}{}
\renewcommand{\Number}{12}\InputImage{\sfl}{\tsc}{\sfl}{\tsc}{\sfl}{\tsc}
{}{}

%%%
\clearpage
\renewcommand{\mat}{Corduroy}
\subsection{\mat}

\renewcommand{\Number}{01}\InputImage{\sexor}{\tsm}{\sexor}{\tsm}{\sexor}{\tbu}
{Logo in the right bottom corner.}{}
\renewcommand{\Number}{02}\InputImage{\sfl}{\tve}{\sexor}{\tbu}{\sexdi}{\tbu}
{}{Considering that the perpendicular direction of the lines makes them disorganized.}
\renewcommand{\Number}{03}\InputImage{\sfl}{\tsm}{\sexor}{\tbu}{\sexdi}{\tbu}
{}{}
\renewcommand{\Number}{04}\InputImage{\sfl}{\tsm}{\sexor}{\tbu}{\sexdi}{\tbu}
{}{}
\renewcommand{\Number}{05}\InputImage{\sfl}{\tve}{\sexor}{\tbu}{\sexdi}{\tbu}
{}{}
\renewcommand{\Number}{06}\InputImage{\sfl}{\tve}{\sexor}{\tbu}{\sro}{\tbu}
{}{Considering the buttons as part of the course scale.}
\renewcommand{\Number}{07}\InputImage{\sfl}{\tve}{\sexor}{\tbu}{\sexor}{\tbu}
{}{}
\renewcommand{\Number}{08}\InputImage{\sfl}{\tve}{\sexor}{\tbu}{\sexor}{\tbu}
{Logo in the bottom left corner.}{}
\renewcommand{\Number}{09}\InputImage{\sfl}{\tve}{\sexor}{\tbu}{\sexor}{\tbu}
{}{}
\renewcommand{\Number}{10}\InputImage{\sexor}{\tbu}{\sexor}{\tbu}{\sro}{\tbu}
{}{Considering the figures of the bottom left corner for the coarse scale.}
\renewcommand{\Number}{11}\InputImage{\sfl}{\tve}{\sexor}{\tbu}{\sexor}{\tbu}
{}{}
\renewcommand{\Number}{12}\InputImage{\sexor}{\tbu}{\sexor}{\tbu}{\sro}{\tbu}
{}{Considering the flower for the coarse scale.}

%%%
\clearpage
\renewcommand{\mat}{Denim}
\subsection{\mat}

\renewcommand{\Number}{01}\InputImage{\sexor}{\tsm}{\sexor}{\tsm}{\sexor}{\tsm}
{}{}
\renewcommand{\Number}{02}\InputImage{\sexor}{\tsm}{\sexor}{\tsm}{\sexor}{\tsm}
{}{}
\renewcommand{\Number}{03}\InputImage{\sexor}{\tsm}{\sexor}{\tsm}{\sexdi}{\tbu}
{}{}
\renewcommand{\Number}{04}\InputImage{\sexor}{\tsm}{\sexor}{\tsm}{\sexdi}{\tbu}
{}{}
\renewcommand{\Number}{05}\InputImage{\sexor}{\tsm}{\sexor}{\tsm}{\sexor}{\tsm}
{}{}
\renewcommand{\Number}{06}\InputImage{\sro}{\tsm}{\sro}{\tsm}{\sro}{\tsm}
{}{}
\renewcommand{\Number}{07}\InputImage{\sexor}{\tsm}{\sexor}{\tsm}{\sexdi}{\tsm}
{}{}
\renewcommand{\Number}{08}\InputImage{\sexor}{\tsm}{\sexor}{\tsm}{\sexor}{\tsm}
{Watermark.}{}
\renewcommand{\Number}{09}\InputImage{\sro}{\tsm}{\sro}{\tsm}{\sro}{\tsm}
{}{}
\renewcommand{\Number}{10}\InputImage{\sexor}{\tsm}{\sexor}{\tsm}{\sexdi}{\tbu}
{}{}
\renewcommand{\Number}{11}\InputImage{\sro}{\tsm}{\sro}{\tsm}{\sro}{\tbu}
{}{}
\renewcommand{\Number}{12}\InputImage{\sro}{\tsm}{\sro}{\tsm}{\sro}{\tbu}
{}{}

%%%
\clearpage
\renewcommand{\mat}{Elm}
\subsection{\mat}

\renewcommand{\Number}{01}\InputImage{\sfl}{\tco}{\sexdi}{\tbu}{\sexdi}{\tbu}
{}{Considering the bark to be disorganized.}
\renewcommand{\Number}{02}\InputImage{\sfl}{\tco}{\sro}{\tbu}{\sro}{\tbu}
{}{}
\renewcommand{\Number}{03}\InputImage{\sfl}{\tco}{\sexdi}{\tbu}{\sexdi}{\tbu}
{}{}
\renewcommand{\Number}{04}\InputImage{\sfl}{\tco}{\sexdi}{\tbu}{\sexdi}{\tbu}
{}{}
\renewcommand{\Number}{05}\InputImage{\sfl}{\tco}{\sexdi}{\tbu}{\sexdi}{\tbu}
{}{}
\renewcommand{\Number}{06}\InputImage{\sfl}{\tco}{\sexor}{\tbu}{\sexor}{\tbu}
{}{Considering the bark as organized.}
\renewcommand{\Number}{07}\InputImage{\sfl}{\tco}{\sexdi}{\tbu}{\sexdi}{\tbu}
{}{}
\renewcommand{\Number}{08}\InputImage{\sfl}{\tco}{\sexdi}{\tbu}{\sexdi}{\tbu}
{}{}
\renewcommand{\Number}{09}\InputImage{\sfl}{\tco}{\sexdi}{\tbu}{\sexdi}{\tbu}
{}{}
\renewcommand{\Number}{10}\InputImage{\sfl}{\tco}{\sfl}{\tco}{\sexdi}{\tbu}
{}{}
\renewcommand{\Number}{11}\InputImage{\sfl}{\tco}{\sexdi}{\tbu}{\sexdi}{\tbu}
{Extremely low quality image.}{}
\renewcommand{\Number}{12}\InputImage{\sfl}{\tco}{\sro}{\tbu}{\sro}{\tbu}
{}{}

%%%
\clearpage
\renewcommand{\mat}{Feathers}
\subsection{\mat}

\renewcommand{\Number}{01}\InputImage{\sexor}{\tbu}{\sexor}{\tbu}{\sro}{\tfe}
{Watermark.}{}
\renewcommand{\Number}{02}\InputImage{\sexor}{\tbu}{\sexor}{\tbu}{\sro}{\tfe} 
{}{}
\renewcommand{\Number}{03}\InputImage{\sexor}{\tbu}{\sexor}{\tbu}{\sro}{\tfe} 
{}{}
\renewcommand{\Number}{04}\InputImage{\sexor}{\tbu}{\sexor}{\tbu}{\sexdi}{\tfe}
{}{}
\renewcommand{\Number}{05}\InputImage{\sexor}{\tbu}{\sro}{\tfe}{\sro}{\tfe}
{Watermark.}{}
\renewcommand{\Number}{06}\InputImage{\sfl}{\tsm}{\sro}{\tfe}{\sro}{\tfe}
{}{}
\renewcommand{\Number}{07}\InputImage{\sexor}{\tbu}{\sexdi}{\tfe}{\sexdi}{\tfe}
{}{}
\renewcommand{\Number}{08}\InputImage{\sexor}{\tbu}{\sexor}{\tbu}{\sexdi}{\tfe}
{}{}
\renewcommand{\Number}{09}\InputImage{\sexor}{\tbu}{\sexor}{\tbu}{\sexdi}{\tfe}
{}{}
\renewcommand{\Number}{10}\InputImage{\sfl}{\tsm}{\sro}{\tfe}{\sro}{\tfe}
{}{Considering the wing for the medium scale.}
\renewcommand{\Number}{11}\InputImage{\sexor}{\tbu}{\sexor}{\tfe}{\sexor}{\tfe}
{}{}
\renewcommand{\Number}{12}\InputImage{\sexor}{\tbu}{\sro}{\tfe}{\sro}{\tfe}
{}{}

%%%
\clearpage
\renewcommand{\mat}{Fur}
\subsection{\mat}
Considering that the pattern that follows the fur is, at least, at low scale, organized.

\renewcommand{\Number}{01}\InputImage{\sexor}{\tfu}{\sexor}{\tfu}{\sexdi}{\tfu}
{Watermark.}{}
\renewcommand{\Number}{02}\InputImage{\sexor}{\tfu}{\sexor}{\tfu}{\sexdi}{\tfu}
{}{}
\renewcommand{\Number}{03}\InputImage{\sexor}{\tfu}{\sexor}{\tfu}{\sro}{\tfu}
{}{Considering the spots as elements of round shape.}
\renewcommand{\Number}{04}\InputImage{\sexor}{\tfu}{\sexor}{\tfu}{\sexor}{\tfu}
{}{}
\renewcommand{\Number}{05}\InputImage{\sexor}{\tfu}{\sexor}{\tfu}{\sexdi}{\tfu}
{}{}
\renewcommand{\Number}{06}\InputImage{\sexor}{\tfu}{\sexor}{\tfu}{\sexdi}{\tfu}
{}{}
\renewcommand{\Number}{07}\InputImage{\sexor}{\tfu}{\sexdi}{\tfu}{\sexdi}{\tfu}
{}{}
\renewcommand{\Number}{08}\InputImage{\sexor}{\tfu}{\sexor}{\tfu}{\sexdi}{\tfu}
{}{}
\renewcommand{\Number}{09}\InputImage{\sexor}{\tfu}{\sexdi}{\tfu}{\sexdi}{\tfu}
{}{}
\renewcommand{\Number}{10}\InputImage{\sexor}{\tfu}{\sexor}{\tfu}{\sexdi}{\tfu}
{}{}
\renewcommand{\Number}{11}\InputImage{\sexor}{\tfu}{\sexor}{\tfu}{\sro}{\tfu}
{Same as Fur 03 with different color.}{}
\renewcommand{\Number}{12}\InputImage{\sexor}{\tfu}{\sexor}{\tfu}{\sexdi}{\tfu}
{}{}

%%%
\clearpage
\renewcommand{\mat}{Hair}
\subsection{\mat}
Considering that the touch of hair is smooth at a high scale.

\renewcommand{\Number}{01}\InputImage{\sexor}{\tbu}{\sexor}{\tsm}{\sexdi}{\tsm}
{Watermark.}{}
\renewcommand{\Number}{02}\InputImage{\sexor}{\tbu}{\sexdi}{\tsm}{\sexdi}{\tsm}
{Watermark.}{}
\renewcommand{\Number}{03}\InputImage{\sexor}{\tbu}{\sexor}{\tsm}{\sexdi}{\tsm}
{}{}
\renewcommand{\Number}{04}\InputImage{\sexor}{\tbu}{\sexor}{\tsm}{\sexdi}{\tsm}
{}{}
\renewcommand{\Number}{05}\InputImage{\sexor}{\tbu}{\sexor}{\tsm}{\sexor}{\tsm}
{}{}
\renewcommand{\Number}{06}\InputImage{\sexor}{\tbu}{\sexor}{\tsm}{\sexor}{\tsm}
{}{}
\renewcommand{\Number}{07}\InputImage{\sexor}{\tbu}{\sexor}{\tsm}{\sexdi}{\tsm}
{}{}
\renewcommand{\Number}{08}\InputImage{\sexor}{\tbu}{\sexdi}{\tsm}{\sexdi}{\tsm}
{}{}
\renewcommand{\Number}{09}\InputImage{\sexor}{\tbu}{\sexor}{\tsm}{\sexdi}{\tsm}
{}{}
\renewcommand{\Number}{10}\InputImage{\sexor}{\tbu}{\sexor}{\tsm}{\sexdi}{\tsm}
{Watermark.}{}
\renewcommand{\Number}{11}\InputImage{\sexor}{\tbu}{\sexor}{\tsm}{\sexor}{\tsm}
{}{}
\renewcommand{\Number}{12}\InputImage{\sexor}{\tbu}{\sexor}{\tsm}{\sexdi}{\tsm}
{}{}

%%%
\clearpage
\renewcommand{\mat}{KnitAran}
\subsection{\mat}
Considering that non-linear patterns are disorganized.

\renewcommand{\Number}{01}\InputImage{\sexor}{\tco}{\sexdi}{\tbu}{\sexdi}{\tbu}
{}{}
\renewcommand{\Number}{02}\InputImage{\sexdi}{\tco}{\sexdi}{\tbu}{\sexdi}{\tbu}
{}{}
\renewcommand{\Number}{03}\InputImage{\sexor}{\tco}{\sexdi}{\tbu}{\sexdi}{\tbu}
{}{Considering that the round parts of the pattern are extended disorganized.}
\renewcommand{\Number}{04}\InputImage{\sexor}{\tco}{\sexdi}{\tbu}{\sro}{\tbu}
{}{Considering the buttons at the coarse level.}
\renewcommand{\Number}{05}\InputImage{\sexor}{\tco}{\sexdi}{\tbu}{\sro}{\tbu}
{}{}
\renewcommand{\Number}{06}\InputImage{\sexor}{\tco}{\sexdi}{\tbu}{\sro}{\tbu}
{}{}
\renewcommand{\Number}{07}\InputImage{\sexor}{\tco}{\sexor}{\tco}{\sexdi}{\tbu}
{}{}
\renewcommand{\Number}{08}\InputImage{\sexdi}{\tco}{\sexdi}{\tco}{\sexdi}{\tbu}
{}{}
\renewcommand{\Number}{09}\InputImage{\sexor}{\tco}{\sexdi}{\tbu}{\sexdi}{\tbu}
{}{}
\renewcommand{\Number}{10}\InputImage{\sexor}{\tco}{\sexor}{\tco}{\sexdi}{\tbu}
{}{}
\renewcommand{\Number}{11}\InputImage{\sexor}{\tco}{\sexdi}{\tbu}{\sro}{\tbu}
{}{Considering the buttons at the coarse level.}
\renewcommand{\Number}{12}\InputImage{\sexor}{\tco}{\sexdi}{\tbu}{\sexdi}{\tbu}
{}{}

%%%
\clearpage
\renewcommand{\mat}{KnitGuernsey}
\subsection{\mat}
Considering that non-linear patterns are disorganized.

\renewcommand{\Number}{01}\InputImage{\sexor}{\tco}{\sexor}{\tco}{\sexdi}{\tbu}
{}{}
\renewcommand{\Number}{02}\InputImage{\sexor}{\tco}{\sexor}{\tco}{\sexdi}{\tco}
{}{Ignoring the table behind.}
\renewcommand{\Number}{03}\InputImage{\sro}{\tco}{\sro}{\tco}{\sexdi}{\tbu}
{}{}
\renewcommand{\Number}{04}\InputImage{\sro}{\tco}{\sro}{\tco}{\sexor}{\tbu}
{}{}
\renewcommand{\Number}{05}\InputImage{\sexor}{\tco}{\sexor}{\tco}{\sexdi}{\tbu}
{}{}
\renewcommand{\Number}{06}\InputImage{\sexdi}{\tco}{\sexdi}{\tco}{\sexdi}{\tbu}
{}{}
\renewcommand{\Number}{07}\InputImage{\sro}{\tco}{\sexdi}{\tco}{\sexdi}{\tbu}
{}{}
\renewcommand{\Number}{08}\InputImage{\sexor}{\tco}{\sexor}{\tco}{\sexdi}{\tbu}
{}{}
\renewcommand{\Number}{09}\InputImage{\sro}{\tco}{\sro}{\tco}{\sexdi}{\tbu}
{}{}
\renewcommand{\Number}{10}\InputImage{\sro}{\tco}{\sro}{\tco}{\sexdi}{\tbu}
{}{}
\renewcommand{\Number}{11}\InputImage{\sro}{\tco}{\sro}{\tco}{\sexdi}{\tbu}
{}{}
\renewcommand{\Number}{12}\InputImage{\sro}{\tco}{\sexdi}{\tco}{\sexdi}{\tbu}
{}{}

%%%
\clearpage
\renewcommand{\mat}{Leather}
\subsection{\mat}

\renewcommand{\Number}{01}\InputImage{\sro}{\tsm}{\sro}{\tbu}{\sro}{\tbu}
{}{}
\renewcommand{\Number}{02}\InputImage{\sro}{\tsm}{\sro}{\tbu}{\sro}{\tbu}
{}{}
\renewcommand{\Number}{03}\InputImage{\sro}{\tco}{\sro}{\tbu}{\sro}{\tbu}
{}{}
\renewcommand{\Number}{04}\InputImage{\sro}{\tsm}{\sro}{\tbu}{\sro}{\tbu}
{}{}
\renewcommand{\Number}{05}\InputImage{\sro}{\tsm}{\sro}{\tbu}{\sro}{\tbu}
{}{Same as leather 04.}
\renewcommand{\Number}{06}\InputImage{\sro}{\tsm}{\sro}{\tbu}{\sro}{\tbu}
{}{}
\renewcommand{\Number}{07}\InputImage{\sro}{\tsm}{\sro}{\tbu}{\sro}{\tbu}
{}{}
\renewcommand{\Number}{08}\InputImage{\sro}{\tsm}{\sro}{\tbu}{\sro}{\tbu}
{}{}
\renewcommand{\Number}{09}\InputImage{\sro}{\tsm}{\sro}{\tbu}{\sro}{\tbu}
{}{}
\renewcommand{\Number}{10}\InputImage{\sro}{\tsm}{\sro}{\tbu}{\sro}{\tbu}
{}{}
\renewcommand{\Number}{11}\InputImage{\sro}{\tsm}{\sro}{\tbu}{\sexor}{\tbu}
{}{}
\renewcommand{\Number}{12}\InputImage{\sro}{\tco}{\sro}{\tbu}{\sro}{\tbu}
{}{}

%%%
\clearpage
\renewcommand{\mat}{Marble}
\subsection{\mat}

\renewcommand{\Number}{01}\InputImage{\sfl}{\tsm}{\sexdi}{\tsm}{\sexdi}{\tsm}
{}{}
\renewcommand{\Number}{02}\InputImage{\sfl}{\tsm}{\sexdi}{\tsm}{\sexdi}{\tsm}
{}{}
\renewcommand{\Number}{03}\InputImage{\sfl}{\tco}{\sfl}{\tco}{\sexdi}{\tco}
{}{}
\renewcommand{\Number}{04}\InputImage{\sfl}{\tco}{\sexdi}{\tco}{\sexdi}{\tco}
{}{}
\renewcommand{\Number}{05}\InputImage{\sfl}{\tsm}{\sexdi}{\tco}{\sexdi}{\tco}
{}{}
\renewcommand{\Number}{06}\InputImage{\sexdi}{\tco}{\sexdi}{\tco}{\sexdi}{\tco}
{}{}
\renewcommand{\Number}{07}\InputImage{\sexdi}{\tco}{\sexdi}{\tco}{\sexdi}{\tco}
{}{}
\renewcommand{\Number}{08}\InputImage{\sexdi}{\tsm}{\sexdi}{\tsm}{\sexdi}{\tsm}
{}{}
\renewcommand{\Number}{09}\InputImage{\sro}{\tsm}{\sro}{\tsm}{\sro}{\tsm}
{}{}
\renewcommand{\Number}{10}\InputImage{\sfl}{\tsm}{\sexdi}{\tsm}{\sro}{\tsm}
{}{}
\renewcommand{\Number}{11}\InputImage{\sro}{\tco}{\sro}{\tco}{\sro}{\tco}
{}{}
\renewcommand{\Number}{12}\InputImage{\sfl}{\tsm}{\sexdi}{\tsm}{\sexdi}{\tsm}
{}{}

%%%
\clearpage
\renewcommand{\mat}{Scale}
\subsection{\mat}

\renewcommand{\Number}{01}\InputImage{\sfl}{\tsm}{\sro}{\tbu}{\sro}{\tbu}
{}{}
\renewcommand{\Number}{02}\InputImage{\sfl}{\tco}{\sro}{\tbu}{\sro}{\tbu}
{}{}
\renewcommand{\Number}{03}\InputImage{\sfl}{\tsm}{\sro}{\tbu}{\sro}{\tbu}
{}{}
\renewcommand{\Number}{04}\InputImage{\sfl}{\tsm}{\sro}{\tbu}{\sro}{\tbu}
{}{}
\renewcommand{\Number}{05}\InputImage{\sfl}{\tsm}{\sro}{\tbu}{\sro}{\tbu}
{}{}
\renewcommand{\Number}{06}\InputImage{\sfl}{\tsm}{\sro}{\tbu}{\sro}{\tbu}
{Watermark.}{}
\renewcommand{\Number}{07}\InputImage{\sfl}{\tsm}{\sro}{\tbu}{\sro}{\tbu}
{Watermark.}{}
\renewcommand{\Number}{08}\InputImage{\sfl}{\tsm}{\sro}{\tbu}{\sro}{\tbu}
{}{}
\renewcommand{\Number}{09}\InputImage{\sfl}{\tsm}{\sro}{\tbu}{\sro}{\tbu}
{}{}
\renewcommand{\Number}{10}\InputImage{\sfl}{\tsm}{\sro}{\tbu}{\sro}{\tbu}
{}{}
\renewcommand{\Number}{11}\InputImage{\sfl}{\tsm}{\sro}{\tbu}{\sro}{\tbu}
{}{}
\renewcommand{\Number}{12}\InputImage{\sfl}{\tsm}{\sro}{\tbu}{\sro}{\tbu}
{}{}

%%%
\clearpage
\renewcommand{\mat}{Silk}
\subsection{\mat}

\renewcommand{\Number}{01}\InputImage{\sfl}{\tsm}{\sfl}{\tsm}{\sexdi}{\tbu}
{}{}
\renewcommand{\Number}{02}\InputImage{\sfl}{\tsm}{\sfl}{\tsm}{\sexdi}{\tbu}
{}{}
\renewcommand{\Number}{03}\InputImage{\sfl}{\tsm}{\sfl}{\tsm}{\sexdi}{\tbu}
{}{}
\renewcommand{\Number}{04}\InputImage{\sexor}{\tsm}{\sfl}{\tsm}{\sexdi}{\tbu}
{}{}
\renewcommand{\Number}{05}\InputImage{\sfl}{\tsm}{\sfl}{\tsm}{\sexdi}{\tbu}
{}{}
\renewcommand{\Number}{06}\InputImage{\sfl}{\tsm}{\sfl}{\tsm}{\sexdi}{\tbu}
{}{}
\renewcommand{\Number}{07}\InputImage{\sfl}{\tsm}{\sfl}{\tsm}{\sexdi}{\tbu}
{}{}
\renewcommand{\Number}{08}\InputImage{\sfl}{\tsm}{\sfl}{\tsm}{\sexdi}{\tbu}
{}{}
\renewcommand{\Number}{09}\InputImage{\sfl}{\tsm}{\sfl}{\tsm}{\sexdi}{\tbu}
{}{}
\renewcommand{\Number}{10}\InputImage{\sexor}{\tsm}{\sfl}{\tsm}{\sexdi}{\tbu}
{}{}
\renewcommand{\Number}{11}\InputImage{\sfl}{\tsm}{\sfl}{\tsm}{\sexdi}{\tbu}
{}{}
\renewcommand{\Number}{12}\InputImage{\sfl}{\tsm}{\sfl}{\tsm}{\sexdi}{\tbu}
{}{}

%%%
\clearpage
\renewcommand{\mat}{Slate}
\subsection{\mat}

\renewcommand{\Number}{01}\InputImage{\sfl}{\tco}{\sexdi}{\tco}{\sexdi}{\tco}
{}{}
\renewcommand{\Number}{02}\InputImage{\sfl}{\tco}{\sexor}{\tco}{\sexor}{\tco}
{?}{}
\renewcommand{\Number}{03}\InputImage{\sfl}{\tco}{\sexdi}{\tco}{\sexdi}{\tco}
{}{}
\renewcommand{\Number}{04}\InputImage{\sfl}{\tco}{\sfl}{\tco}{\sexdi}{\tco}
{}{}
\renewcommand{\Number}{05}\InputImage{\sfl}{\tco}{\sexdi}{\tco}{\sexdi}{\tco}
{}{}
\renewcommand{\Number}{06}\InputImage{\sfl}{\tco}{\sexdi}{\tco}{\sexdi}{\tco}
{}{}
\renewcommand{\Number}{07}\InputImage{\sfl}{\tco}{\sexdi}{\tco}{\sexdi}{\tbu}
{}{}
\renewcommand{\Number}{08}\InputImage{\sfl}{\tco}{\sfl}{\tco}{\sfl}{\tco}
{}{}
\renewcommand{\Number}{09}\InputImage{\sfl}{\tco}{\sexdi}{\tco}{\sexdi}{\tco}
{}{}
\renewcommand{\Number}{10}\InputImage{\sfl}{\tco}{\sexdi}{\tco}{\sexdi}{\tco}
{}{}
\renewcommand{\Number}{11}\InputImage{\sfl}{\tco}{\sexdi}{\tco}{\sexdi}{\tco}
{Watermark.}{Same as slate 09.}
\renewcommand{\Number}{12}\InputImage{\sfl}{\tco}{\sexdi}{\tco}{\sexdi}{\tco}
{}{}

%%%
\clearpage
\renewcommand{\mat}{Stucco}
\subsection{\mat}

\renewcommand{\Number}{01}\InputImage{\sfl}{\tsm}{\sro}{\tco}{\sro}{\tco}
{}{}
\renewcommand{\Number}{02}\InputImage{\sro}{\tco}{\sro}{\tco}{\sro}{\tco}
{Watermark.}{}
\renewcommand{\Number}{03}\InputImage{\sfl}{\tco}{\sfl}{\tco}{\sfl}{\tco}
{}{}
\renewcommand{\Number}{04}\InputImage{\sfl}{\tco}{\sfl}{\tco}{\sfl}{\tco}
{}{}
\renewcommand{\Number}{05}\InputImage{\sfl}{\tco}{\sro}{\tco}{\sro}{\tco}
{}{}
\renewcommand{\Number}{06}\InputImage{\sfl}{\tco}{\sro}{\tco}{\sro}{\tco}
{}{}
\renewcommand{\Number}{07}\InputImage{\sfl}{\tco}{\sfl}{\tco}{\sexdi}{\tbu}
{}{}
\renewcommand{\Number}{08}\InputImage{\sfl}{\tco}{\sfl}{\tco}{\sfl}{\tco}
{}{}
\renewcommand{\Number}{09}\InputImage{\sfl}{\tco}{\sfl}{\tco}{\sfl}{\tco}
{}{}
\renewcommand{\Number}{10}\InputImage{\sfl}{\tco}{\sro}{\tco}{\sro}{\tco}
{}{Same as stucco 05.}
\renewcommand{\Number}{11}\InputImage{\sfl}{\tco}{\sro}{\tco}{\sro}{\tco}
{}{}
\renewcommand{\Number}{12}\InputImage{\sfl}{\tco}{\sfl}{\tco}{\sfl}{\tco}
{}{}

%%%
\clearpage
\renewcommand{\mat}{Velour}
\subsection{\mat}

\renewcommand{\Number}{01}\InputImage{\sfl}{\tve}{\sfl}{\tve}{\sfl}{\tve}
{}{}
\renewcommand{\Number}{02}\InputImage{\sfl}{\tve}{\sfl}{\tve}{\sexdi}{\tbu}
{}{}
\renewcommand{\Number}{03}\InputImage{\sfl}{\tve}{\sfl}{\tve}{\sexdi}{\tbu}
{}{}
\renewcommand{\Number}{04}\InputImage{\sfl}{\tve}{\sfl}{\tve}{\sexdi}{\tbu}
{}{}
\renewcommand{\Number}{05}\InputImage{\sexdi}{\tco}{\sexdi}{\tco}{\sfl}{\tve}
{}{}
\renewcommand{\Number}{06}\InputImage{\sfl}{\tve}{\sfl}{\tve}{\sexor}{\tbu}
{}{}
\renewcommand{\Number}{07}\InputImage{\sfl}{\tve}{\sfl}{\tve}{\sexdi}{\tbu}
{}{}
\renewcommand{\Number}{08}\InputImage{\sfl}{\tve}{\sfl}{\tve}{\sexdi}{\tbu}
{}{}
\renewcommand{\Number}{09}\InputImage{\sfl}{\tve}{\sfl}{\tve}{\sexdi}{\tbu}
{}{}
\renewcommand{\Number}{10}\InputImage{\sfl}{\tve}{\sfl}{\tve}{\sexdi}{\tbu}
{}{}
\renewcommand{\Number}{11}\InputImage{\sfl}{\tve}{\sfl}{\tve}{\sexor}{\tbu}
{}{}
\renewcommand{\Number}{12}\InputImage{\sfl}{\tve}{\sfl}{\tve}{\sfl}{\tve}
{}{}


\end{appendix}

\end{singlespace}
\end{document}