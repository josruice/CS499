%!TEX root = final_report.tex

\section{Engineering}

\subsection{Tools}
The main tool used while completing this project has been \emph{MATLAB}. Its  powerful built-in functions and routines for vector and matrix manipulation, along with its wide popularity in academia and, in particular, among the image processing scientists, made it really suitable for the purpose. All the code has been written in \emph{MATLAB} programming language and the debugger that comes with its \emph{IDE} has been widely used to test and polish it. \\

The text editor used to write all the code has been \emph{Sublime Text 2}, because of its flexibility, \emph{MATLAB} syntax highlight and features like multiple cursors or quick selection of the same word, which make the coding task much easier and smooth. \\

This report has been generated using \LaTeX. The main reason to choose it over a \emph{WYSIWYG} editor are the look and feel of the result and its flexibility in many aspects, like the definition of reusable functions or the availability of packages for almost any imaginable task. 


\subsection{Libraries}
\emph{VLFeat} has been the main library for this project. Its functions for \emph{SIFT}, \emph{k-means} and \emph{SVMs} have been really helpful for the implementation. For a complete reference about the library, \cite{Vedaldi_2010_VLFEAT} should be checked. 


\subsection{Maintenance}
Every part of this project has been included in the version-control system \emph{Git} since the beginning, and uploaded frequently to \emph{Github}\footnote{\url{https://github.com/josruice/CS499}}, a website where the code is publicly available. \\

Besides, \emph{MATLAB} code has been written taking into account one of the most important styleguides of the language \cite{Johnson_2002_MPS}, to produce code cleaner and easier to read and understand. Also, comments have been included to emphasize even more the later point. 


% \subsection{Other technical details}

