%!TEX root = final_report.tex

\section{Conclusions and future work} \label{sec:conclusions}

Material recognition using \emph{Computer Vision} is still an open problem in the industry. Many different approaches have been tried to solve it, being \emph{Deep Learning} the one that, nowadays, is giving the most promising results. \\

Even though, it is still worth it to try different, new approaches. In this work, we have covered a solution based on classic \emph{Machine Learning} techniques, such as \emph{SVMs}, \emph{k-means} and \emph{Naive Bayes}, along with the powerful and widely used \emph{SIFT} descriptors, but including an innovation in the use of an intermediate step based on a vocabulary of material properties. \\

We have analyzed the hightlights and lowlights of the approach, both from a theoretical and practical point of view, providing results using several metrics for different parts of the algorithm. \\

While the results look promising when using the real material properties to classify materials, further research is required to test the effects of a higher dataset and different feature descriptors methods in terms of accuracy. Even if better results are achieved in harsher conditions, the high cost of the manual markup of properties is still an important drawback of the algorithm. \\

Apart from the material recognition use, it cannot be denied that the information obtained in the intermediate step provides value to the classification process that could be useful even in real life applications. Besides, the theoretical possibility of one sample classification is interesting enough to motivate the continuation of this work.