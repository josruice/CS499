%!TEX root = final_report.tex

\section{Introduction}

The word \emph{material} has its origin in the \emph{Latin} word `\emph{materia}', in English, \emph{matter}. In the same way that reality is made of \emph{matter}, every concrete entity can be defined in terms of materials: the silk of a dress, the scales of a python or the concrete of the pavement, for example. Therefore, materials represent an important part of the real world, since they are the key to understand and infer the physical properties of objects just looking at them. \\

Human beings, thanks to thousand of years of evolution, have become very good at recognizing materials. This skill could even determine the difference between life and death in the beginning of times, when being able to start a fire depended on choosing the right type of rock to make sparks. Nowadays, fortunately, our life does not depend on this ability, but we still use it everyday, even without noticing, for example, when deciding which side of a sidewalk step to avoid slippery ice during winter. \\

Because of all of this, an automatic system able to see the objects and recognize their materials would be really valuable, since it would be one more step on the way of building a machine able to see as well as a human, or even better. In that sense, despite of the achievements of \emph{Computer Vision}, material recognition is still an open problem in the field. \\

We present a new approach in the problem of materials recognition, using feature extraction methods and \emph{Support Vector Machines} (\emph{SVMs}) to obtain the physical properties of the objects and applying \emph{SVMs} again to this intermediate dataset to obtain the final material category in a set of 18 possible ones. The obtained results using the dataset of \cite{Liao_2013_CVPR} beat any other system used on this dataset and prove that this approach has a great potential as a first step in the solution to the problem.